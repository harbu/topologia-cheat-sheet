\documentclass[landscape,a4paper,10pt]{article}

\usepackage[T1]{fontenc}        % cleaner font
\usepackage[utf8]{inputenc}     % UTF-8 support
\usepackage{amsmath}            % more math symbols
\usepackage{amssymb}            % allow math symbols of form \mathbb{...}
\usepackage{graphicx}           % enhanced graphics support
\usepackage{epstopdf}           % automagically turn eps to pdf, for gnuplot
\usepackage{subcaption}         % subfigure support
\usepackage[dvipsnames]{xcolor} % syntax coloring support
\usepackage{listings}           % programming language support
\usepackage{algorithmic}        % pseudocode support
\usepackage{algorithm}
\usepackage{cancel}             % cancel out terms in division
\usepackage{parskip}            % enable spacing between paragraphs
\usepackage{cases}              % enable math function definitions with cases
\usepackage{titling}            % allow adjustment of document title
\usepackage{fullpage}           % 1 inch margins
\usepackage{bm}                 % bold math fonts for vectors and stuff
\usepackage{nicefrac}           % Nicer looking fractions
\usepackage{floatflt}           % Allows text to float around figures / tables
\usepackage{multicol}
\usepackage[landscape]{geometry}
\usepackage{amsthm}
\usepackage{titlesec}

\usepackage[finnish]{babel}    % use Finnish for hyphenation
%\usepackage{hyperref}          % make clickable urls
%\usepackage{color}             % colored text?

% CHEAT SHEET: minimize margins
\geometry{top=0.5cm,left=0.5cm,right=0.5cm,bottom=0.5cm}

% CHEAT SHEET: turn off header and footer
\pagestyle{empty}

% CHEAT SHEET: Redefine section commands to use less space
\makeatletter
\renewcommand{\section}{\@startsection{section}{1}{0mm}%
                                {-1ex plus -.5ex minus -.2ex}%
                                {0.5ex plus .2ex}%x
                                {\color{blue}\normalfont\large\bfseries}}
\renewcommand{\subsection}{\@startsection{subsection}{2}{0mm}%
                                {-1explus -.5ex minus -.2ex}%
                                {0.5ex plus .2ex}%
                                {\normalfont\normalsize\bfseries}}
\renewcommand{\subsubsection}{\@startsection{subsubsection}{3}{0mm}%
                                {-1ex plus -.5ex minus -.2ex}%
                                {1ex plus .2ex}%
                                {\normalfont\small\bfseries}}
\makeatother

% CHEAT SHEET: Don't print section numbers
\setcounter{secnumdepth}{0}

\setlength{\parindent}{0pt}
\setlength{\parskip}{0pt plus 0.5ex}

% Adjust title vertical position
\setlength{\droptitle}{-2cm}

% Set up syntax highlighting for programming languages
\lstloadlanguages{Ruby}
\lstset{%
basicstyle=\ttfamily\bfseries\footnotesize,
commentstyle = \ttfamily\color{orange},
keywordstyle=\ttfamily\color{blue},
stringstyle=\color{red},
showstringspaces=false,
frame=trbl,
}

% Custom probability macros
\def\ci{\perp\!\!\!\perp}              % Independence symbol
\newcommand{\jpr}[2]{P(#1 \, , \, #2)} % Joint probability
\newcommand{\cpr}[2]{P(#1 \, | \, #2)} % Conditional probability


% Avoin joukko / Suljettu joukko symbolit
\newcommand\opn{\mathrel{\ooalign{$\subset$\cr
  \hidewidth\hbox{$\circ\mkern.5mu$}\cr}}}
\newcommand\cls{\mathrel{\ooalign{$\subset$\cr
\hidewidth\raise.225ex\hbox{$\text{{\scriptsize c}}\mkern2mu$}\cr}}}

% Theorems
\newtheoremstyle{customtheoremstyle}{}{}{}{}{\bfseries}{}{ }{}
\theoremstyle{customtheoremstyle}
\newtheorem*{theorem}{Lause}
\newtheorem*{definition}{Määritelmä}
\newtheorem*{example}{Esim.}

\begin{document}
\begin{multicols*}{3}
% CHEAT SHEET: Multicol params

% --------- Esitietoja ---------%
\section{Esitietoja}

\begin{theorem}
  $A_1 \subset A_2 \iff {A_2}^C \subset {A_1}^C$
\end{theorem}

\begin{theorem}
  Olkoon $x,y \in \mathbb{R}$.
  \begin{align*}
    \max \{x,y\} = \nicefrac{1}{2}(x + y + |x-y|) \\
    \min \{x,y\} = \nicefrac{1}{2}(x + y - |x-y|)
  \end{align*}
\end{theorem}

% --------- Vektoriavaruudet ---------%
\section{Vektoriavaruudet}

\newtheorem*{vktava}{Vektoriavaruus}
\begin{vktava}
  on joukko $E$ jossa $\forall x,y \in E$ ja $\forall a,b \in \mathbb{R}$
  \begin{multicols}{2}
    \begin{itemize}
      \item{$x + y \in E$, $ax \in E$}
      \item{$x + y$ vaihd. ja liitän.}
      \item{$\exists \in \overline0 \; \text{s.e.} \; \overline0 + x = x$}
      \item{$\exists -x \in E$: $x + (-x) = \overline0$}
      \item{$(ab)x = a(bx)$}
      \item{$a(x + y) = ax + ay$}
      \item{$(a + b)x = ax + bx$}
      \item{$1x = x$}
    \end{itemize}
  \end{multicols}
\end{vktava}

\newtheorem*{sistuloava}{Sisätuloavaruus}
\begin{sistuloava}
  on vektoriavaruus $E$ sekä sisätulo $(x,y) \mapsto x \cdot y : E \times E
  \rightarrow \mathbb{R}$, jossa $\forall x,y,z \in E$ ja
  $\forall a \in \mathbb{R}$ pätee
  \begin{multicols}{2}
    \begin{enumerate}
      \item[(S1)]{$x \cdot y = y \cdot x$}
      \item[(S2)]{$(ax) \cdot y = a(x \cdot y)$}
      \item[(S3)]{$(x+y) \cdot z = x \cdot z + y \cdot z$}
      \item[(S4)]{$x \cdot x \geq 0$}
      \item[(S5)]{$x \cdot x = 0 \iff x = \overline{0}$}
    \end{enumerate}
  \end{multicols}
\end{sistuloava}

\newtheorem*{normi}{Normiavaruus}
\begin{normi}
  on vektoriavaruus $E$ sekä normi
  $x \mapsto \|x\| : E \rightarrow \mathbb{R}_+$
  \begin{itemize}
    \item[(N1)]{$\|x + y\| \leq \|x\| + \|y\|$}
    \item[(N2)]{$\|ax\| = |a|\|x\|$}
    \item[(N3)]{$\|x\| = 0 \iff x = \overline0$}
  \end{itemize}
\end{normi}

\begin{theorem}
  Jokainen sisätuloavaruus on normiavaruus.
\end{theorem}

\newtheorem*{schwarz}{Schwarzin ey}
\begin{schwarz}
  $| x \cdot y | \leq \|x\|\|y\|$, missä $x,y \in E$ (sis.tul.ava)
\end{schwarz}

% --------- Metriikka ---------%
\section{Metriikka}
Olkoon $d: X \times X \rightarrow \mathbb{R}_+$.
\begin{definition}
  Kuvaus $d$ on metriikka $X$:ssä, jos $\forall x,y,z \in X$:
  \begin{enumerate}
    \item[(M1)]{$d(x,z) \leq d(x,y) + d(y,z)$}
    \item[(M2)]{$d(x,y) = d(y,x)$}
    \item[(M3)]{$d(x,y) = 0 \iff x = y$}
  \end{enumerate}
\end{definition}

\begin{theorem}
  Kun $X := (E,\|.\|)$ saadaan normin määräämä metriikka $d(x,y) = \| x - y\|$.
\end{theorem}

% TODO: kuulat / pallot

\newtheorem*{setdist}{Joukkojen välinen etäisyys}
\begin{setdist}
  Olk. $(X,d)$ M.A. ja $\emptyset \neq A \subset X$, $\emptyset \neq B
  \subset X$. Nyt $d(A,B) = \inf \{ d(x,y) \; | \; x \in A, y \in B \}$.
  Erikoistapaus $B = \{z\}$ (yksiö).
\end{setdist}

\begin{theorem}
  $| d(x,A) - d(y,A) | \leq d(x,y)$
\end{theorem}

\newtheorem*{läpimitta}{Joukon läpimitta}
\begin{läpimitta}
  Olk. $(X,d)$ M.A. ja $\emptyset \neq A \subset X$. Nyt
  $d(A) = \sup \{d(x,y) \; | \; x \in A, y \in A\} \in R_+ \cup \{\infty\}$.
  $d(\emptyset) = 0$.
\end{läpimitta}

\begin{theorem}
  Jos $A \subset B \subset X$, niin $d(A) \leq d(B)$.
\end{theorem}

% TODO: d(A) = 0 <=> #A <= 1 (?)

\begin{theorem}
  Olk. $(X,d)$ M.A., $a \in X$, $r > 0$. Nyt $d(B(a,r)) \leq d(\bar{B}(a,r))
  \leq 2r$. Jos $X \neq \{\overline0\}$ normiavaruus, niin $=$.
\end{theorem}

\newtheorem*{rajoitetut}{Rajoitetut joukot}
\begin{rajoitetut}
  Osajoukko $A \subset (X,d)$ on raj., jos $d(A) < \infty$.
\end{rajoitetut}
% TODO: rajoitettu kuvaus (?)

\begin{theorem}
  Olk. $(X,d)$ M.A., $X \neq \emptyset$. Joukko $A \subset X$ on raj., $\iff$
  se sisältyy $X$:n johonkin kuulaan $B(a,r)$.
\end{theorem}

\begin{theorem}
  $A$ ja $B$ ovat $(X,d)$ raj. osajoukkoja $\implies$ $A \cup B$ raj. Jos
  lisäksi $A \neq \emptyset \neq B$ niin $d(A \cup B) \leq d(A) + d(B) +
  d(A,B)$.
\end{theorem}

% --------- Avoimet joukot ---------%
\section{Avoimet joukot}
Olkoon $(X,d)$ metrinen avaruus (=MA)
\begin{definition}
  $U \opn X \iff \forall x \in U \; \exists B(x,r) \subset U \quad (r>0)$
\end{definition}

\begin{theorem}
  Kun kaikki $U_j \opn X$ niin
  \begin{enumerate}
    \item{$\cup_{j \in J} U_j \opn X$.}
    \item{$\cap_{i=1}^{n} U_j \opn X$.}
    \item{$X \opn X$, $\emptyset \opn X$.}
  \end{enumerate}
\end{theorem}

\begin{theorem}
$U \opn X \iff$ $U = \emptyset$ tai $U = \bigcup \{B(x,r(x)) : x \in U\}$
\end{theorem}

\newtheorem*{ystö}{Ympäristö}
\begin{ystö}
  Joukko $U \opn X$ on pisteen $a \in X$ ystö, jos $a \in U$. Jos $A \subset
  U$, niin $U$ on joukon $A$ ystö.
\end{ystö}

\begin{theorem}
  $A \opn X \iff \forall x \in A \: \exists U(x) \subset A$ missä $U(x)$
  ystöjä.
\end{theorem}

\begin{theorem}
  Jos $a,b \in X, \; a \neq b$, niin niillä on erilliset ystöt.
\end{theorem}

\newtheorem*{erakkopiste}{Erakkopiste}
\begin{erakkopiste}
 Piste $a \in A$ on $A \subset X$:n erakkopiste jos $\exists$ $a$:n ystö $V$ s.e. $V \cap A = \{a\}$.
\end{erakkopiste}

\newtheorem*{diskjoukko}{Diskreetti joukko}
\begin{diskjoukko}
  on $A$ jos sen kaikki pisteet ovat erakkoja.
\end{diskjoukko}

\begin{theorem}
  Avaruus $X$ diskreetti $\iff$ kaikki osajoukot avoimia.
\end{theorem}

\begin{example}
  äärellinen joukko ja $\mathbb{Z}$ diskreettejä. Myös $\{0,1\}$-metriikka
  tekee diskreetin.
\end{example}

% --------- Jatkuvuus ---------%
\section{Jatkuvuus}
Olkoon $(X,d)$, $(Y,d')$ M.A ja $(E,\|.\|)$ normiavaruus.

\begin{definition}
  $f: X \rightarrow Y$ on jva pisteessä $a$, jos $\forall \varepsilon > 0 \;
  \exists \delta > 0$ s.e. $d'(f(x),f(a)) < \varepsilon$ aina kun $x \in X \:
  \land \: d(x,a) < \delta$. Ehto voidaan kirjoittaa muodossa
  $f B(a,\delta) \subset B(f(a), \varepsilon)$ tai $B(a,\delta) \subset
  f^{-1} B(f(a),\epsilon)$.
\end{definition}

\newtheorem*{lipschitz}{Lipschitz-kuvaus}
\begin{lipschitz}
  $f: X \rightarrow Y$ on $M$-Lipschitz (eli myös jva) jos $\exists M \geq 0$ s.e.
  $d'(f(x),f(y)) \leq M d(x,y), \; \forall x,y \in X$.
\end{lipschitz}

\begin{example}
  $f: X \rightarrow \mathbb{R}, \: f(x) = d(x,A)$ on $1$-Lipschitz.
\end{example}

\begin{theorem}
  S.E.Y.
  \begin{enumerate}
    \item[(1)]{$f$ jva pisteessä $a$.}
    \item[(2)]{$\forall$ $f(a)$ ystöä $V$ kohti $\exists$ $a$:n ystö U s.e. $f
      U \subset V$.}
    \item[(3)]{$\forall$ $f(a)$ ystöä $V$ kohti $\exists$ $a$:n ystö U s.e. $U
      \subset f^{-1} V$.}
  \end{enumerate}
\end{theorem}

\newtheorem*{alkukuvaehto}{Alkukuvaehto}
\begin{alkukuvaehto}
  $f: X \rightarrow Y$ jva $\iff$ $\forall V \opn Y$: $f^{-1} V \opn X$
\end{alkukuvaehto}

\newtheorem*{rystö}{r-ystö}
\begin{rystö}
  Joukon $\emptyset \neq A \subset X$ $r$-ystö (avoin joukko) on
  $B(A,r) = \{x \in X \; | \; d(x,A) < r\} \quad (r>0)$.
\end{rystö}

\begin{theorem}
  \begin{enumerate}
    \item[(1)]{$X \stackrel{f}{\rightarrow} Y \stackrel{g}{\rightarrow} Z$ jvia
      $\implies$ $g \circ f$ jva}
    \item[(2)]{$f : X \rightarrow \mathbb{R}$ jva ja $f(x) \neq 0$ $\implies$
      $1/f : X \rightarrow \mathbb{R}$ jva}
    \item[(3)]{Jos $f,g: X \rightarrow E$ jvia pisteessä $a$, niin $f+g$ jva
      $a$:ssa.}
    \item[(4)]{Jos $f: X \rightarrow E$ ja $\alpha: X \rightarrow \mathbb{R}$
      jvia pisteessä $a$, niin $\alpha f$ jva $a$:ssa.}
  \end{enumerate}
\end{theorem}

\newtheorem*{projektio}{$\mathbb{R}^n$:n projektiot}
\begin{projektio}
  (jvia) kuvaksia $\text{pr}_j : \mathbb{R}^n \rightarrow \mathbb{R},
  \: \text{pr}_j(x) = x_j$
\end{projektio}

\newtheorem*{komponenttikuvaus}{Komponenttikuvaus}
\begin{komponenttikuvaus}
  Olk. $f: X \rightarrow \mathbb{R}^n$ ehdolla $f(x) = (f_1(x), ...,
  f_n(x)) \in \mathbb{R}^n$. Nyt kaikki $f_j : X \rightarrow \mathbb{R}$ ovat
  komp.kuvauksia. Huom: $f_j = \text{pr}_j \circ f$ ja $f = (f_1,...,f_n)$.
\end{komponenttikuvaus}

\begin{theorem}
  $f: X \rightarrow \mathbb{R}^n$ jva pisteessä $a \in X$ $\iff$ jokainen
  komponenttikuvaus $f_j: X \rightarrow \mathbb{R}$ jva $a$:ssa.
\end{theorem}


% --------- Suljetut joukot ---------%
\section{Suljetut joukot}
\newtheorem*{defn:closedSet}{Suljettu joukko}
\begin{defn:closedSet}
  $F \cls X \iff F^C \opn X$.
\end{defn:closedSet}

\newtheorem*{defn:closure}{Sulkeuma}
\begin{defn:closure}
  \begin{align*}
    \bar{A} &=
    \{ x \in X \; | \; \forall \, U(x) \opn X \; \text{pätee} \; U(x) \cap A \neq \emptyset \}
    \\
    &= \{ x \in X \; | \; \forall r > 0 \; \text{pätee} \; B(x,r) \cap A \neq
    \emptyset \}
  \end{align*}
\end{defn:closure}

\begin{theorem}
  Kun kaikki $F_j \cls X$ niin
  \begin{enumerate}
    \item{$\cap_{j \in J} F_j \cls X$.}
    \item{$\cup_{i=1}^{n} F_j \cls X$.}
    \item{$X \cls X$, $\emptyset \cls X$.}
  \end{enumerate}
\end{theorem}

\begin{theorem}
  Jos $U \opn X$ ja $F \cls X$ niin $U \setminus F \opn X$ ja $F \setminus U
  \cls X$.
\end{theorem}

\begin{theorem}
  Olkoot $A,B \subset X$.
  \begin{enumerate}
    \item[(1)]{$A \subset \bar{A}$}
    \item[(2)]{$\bar{A}$ suljettu}
    \item[(3)]{$A \subset B \cls X \implies \bar{A} \subset B$}
    \item[(4)]{$\bar{A}$ on suppein sulj. osajoukko, joka sisältää $A$:n}
    \item[(5)]{$A \subset B \implies \bar{A} \subset \bar{B}$}
    \item[(6)]{$A \cls X \iff A = \bar{A}$}
    \item[(7)]{$\bar{\bar{A}} = \bar{A}$}
    \item[(8)]{$\overline{A \cup B} = \bar{A} \cup \bar{B}$}
    \item[(9)]{$\overline{A \cap B} \subset \bar{A} \cap \bar{B}$}
  \end{enumerate}
\end{theorem}

\begin{theorem}
  Olkoon $\emptyset \neq A \subset R$ ja $\sup A < \infty$. Tällöin $\sup A \in
  \bar{A}$.
\end{theorem}

\begin{theorem}
  $\bar{A} = \{ x \in X \; | \; d(x,A) = 0 \}$ kun $\emptyset \neq A \subset X$.
\end{theorem}

\begin{theorem}
  $f$ jva pisteessä $a$ $\iff$ Jos $A \subset X$ ja $a \in \bar{A}$ niin
  $f(a) \in \overline{fA}$.
\end{theorem}

\begin{theorem}
  S.E.Y.
  \begin{enumerate}
    \item[(1)]{$f$ on jva}
    \item[(2)]{$\forall \, F \cls Y$: $f^{-1}[F] \cls X$ }
    \item[(3)]{$\forall A \subset X$: $f[\bar{A}] \subset \overline{f[A]}$}
  \end{enumerate}
\end{theorem}
\end{multicols*}
\end{document}

% ---- EXAMPLES BELOW ----

% -- Example of figure for imagename.eps
% \begin{figure}
%     \includegraphics[scale=0.5]{imagename}
%     \caption*{Caption be here.}
% \end{figure}

% -- Example of floating figure for imagename.eps
% \begin{floatingfigure}[r]{0.49\textwidth}
%   \includegraphics[scale=0.31]{imagename}
%   \caption*{Caption be here.}
% \end{floatingfigure}

% -- Example of two figures side-by-side
% \begin{figure}[h]
%   \centering
%   \begin{subfigure}{.5\textwidth}
%     \centering
%     \includegraphics[width=.9\linewidth]{fig1.eps}
%     \caption*{Caption here}
%   \end{subfigure}%
%   \begin{subfigure}{.5\textwidth}
%     \centering
%     \includegraphics[width=.9\linewidth]{fig2.eps}
%     \caption*{Caption be here}
%   \end{subfigure}
%   \caption{Shared caption}
%   \label{fig:fighere}
% \end{figure}

% -- Remove pagination
% \thispagestyle{empty}
% \pagestyle{empty}

% -- Example of code listing
%\lstinputlisting[language=Ruby, caption={file.rb}]{./file.rb}

% -- Example of cases-environment
% $$
% f(x) =
% \begin{cases}
%     x^2 & \text{if } x > 0 \\
%     0 & \text{otherwise } \\
% \end{cases}
% $$

% -- Example of multicolumn row in tabular environment
% \multicolumn{3}{|c|}{Cell spanning three columns}
