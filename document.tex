\documentclass[landscape,a4paper,10pt]{article}

\usepackage[T1]{fontenc}        % cleaner font
\usepackage[utf8]{inputenc}     % UTF-8 support
\usepackage{amsmath}            % more math symbols
\usepackage{amssymb}            % allow math symbols of form \mathbb{...}
\usepackage{graphicx}           % enhanced graphics support
\usepackage{epstopdf}           % automagically turn eps to pdf, for gnuplot
\usepackage{subcaption}         % subfigure support
\usepackage[dvipsnames]{xcolor} % syntax coloring support
\usepackage{listings}           % programming language support
\usepackage{algorithmic}        % pseudocode support
\usepackage{algorithm}
\usepackage{cancel}             % cancel out terms in division
\usepackage{parskip}            % enable spacing between paragraphs
\usepackage{cases}              % enable math function definitions with cases
\usepackage{titling}            % allow adjustment of document title
\usepackage{fullpage}           % 1 inch margins
\usepackage{bm}                 % bold math fonts for vectors and stuff
\usepackage{nicefrac}           % Nicer looking fractions
\usepackage{floatflt}           % Allows text to float around figures / tables
\usepackage{multicol}
\usepackage[landscape]{geometry}
\usepackage{amsthm}
\usepackage{titlesec}

\usepackage[finnish]{babel}    % use Finnish for hyphenation
%\usepackage{hyperref}          % make clickable urls
%\usepackage{color}             % colored text?

% CHEAT SHEET: minimize margins
\geometry{top=0.5cm,left=0.5cm,right=0.5cm,bottom=0.5cm}

% CHEAT SHEET: turn off header and footer
\pagestyle{empty}

% CHEAT SHEET: Redefine section commands to use less space
\makeatletter
\renewcommand{\section}{\@startsection{section}{1}{0mm}%
                                {-1ex plus -.5ex minus -.2ex}%
                                {0.5ex plus .2ex}%x
                                {\color{red}\normalfont\large\bfseries}}
\renewcommand{\subsection}{\@startsection{subsection}{2}{0mm}%
                                {-1explus -.5ex minus -.2ex}%
                                {0.5ex plus .2ex}%
                                {\normalfont\normalsize\bfseries}}
\renewcommand{\subsubsection}{\@startsection{subsubsection}{3}{0mm}%
                                {-1ex plus -.5ex minus -.2ex}%
                                {1ex plus .2ex}%
                                {\normalfont\small\bfseries}}
\makeatother

% CHEAT SHEET: Don't print section numbers
\setcounter{secnumdepth}{0}

\setlength{\parindent}{0pt}
\setlength{\parskip}{0pt plus 0.5ex}

% Adjust title vertical position
\setlength{\droptitle}{-2cm}

% Set up syntax highlighting for programming languages
\lstloadlanguages{Ruby}
\lstset{%
basicstyle=\ttfamily\bfseries\footnotesize,
commentstyle = \ttfamily\color{orange},
keywordstyle=\ttfamily\color{blue},
stringstyle=\color{red},
showstringspaces=false,
frame=trbl,
}

% Custom probability macros
\def\ci{\perp\!\!\!\perp}              % Independence symbol
\newcommand{\jpr}[2]{P(#1 \, , \, #2)} % Joint probability
\newcommand{\cpr}[2]{P(#1 \, | \, #2)} % Conditional probability


% Avoin joukko / Suljettu joukko symbolit
\newcommand\opn{\mathrel{\ooalign{$\subset$\cr
  \hidewidth\hbox{$\circ\mkern.5mu$}\cr}}}
\newcommand\cls{\mathrel{\ooalign{$\subset$\cr
\hidewidth\raise.225ex\hbox{$\text{{\scriptsize c}}\mkern2mu$}\cr}}}

% Theorems
\newtheoremstyle{customtheoremstyle}{}{}{}{}{\bfseries}{}{ }{}
\theoremstyle{customtheoremstyle}
\newtheorem*{theorem}{Lause}
\newtheorem*{definition}{Määritelmä}

\begin{document}
\begin{multicols*}{3}
% CHEAT SHEET: Multicol params

% --------- Esitietoja ---------%
\section{Esitietoja}

\begin{theorem}
  $A_1 \subset A_2 \iff {A_2}^C \subset {A_1}^C$
\end{theorem}

\begin{theorem}
  Olkoon $x,y \in \mathbb{R}$.
  \begin{align*}
    \max \{x,y\} = \nicefrac{1}{2}(x + y + |x-y|) \\
    \min \{x,y\} = \nicefrac{1}{2}(x + y - |x-y|)
  \end{align*}
\end{theorem}

% --------- Vektoriavaruudet ---------%
\section{Vektoriavaruudet}

\newtheorem*{vktava}{Vektoriavaruus}
\begin{vktava}
  on joukko $E$ jossa $\forall x,y \in E$ ja $\forall a,b \in \mathbb{R}$
  \begin{multicols}{2}
    \begin{itemize}
      \item{$x + y \in E$, $ax \in E$}
      \item{$x + y$ vaihd. ja liitän.}
      \item{$\exists \in \overline0 \; \text{s.e.} \; \overline0 + x = x$}
      \item{$\exists -x \in E$: $x + (-x) = \overline0$}
      \item{$(ab)x = a(bx)$}
      \item{$a(x + y) = ax + ay$}
      \item{$(a + b)x = ax + bx$}
      \item{$1x = x$}
    \end{itemize}
  \end{multicols}
\end{vktava}

\newtheorem*{normi}{Normi}
\begin{normi}
  on $x \mapsto \|x\| : E \rightarrow \mathbb{R}_+$ vek.avaruudessa $E$ jos
  \begin{itemize}
    \item[(N1)]{$\|x + y\| \leq \|x\| + \|y\|$}
    \item[(N2)]{$\|ax\| = |a|\|x\|$}
    \item[(N3)]{$\|x\| = 0 \iff x = \overline0$}
  \end{itemize}
\end{normi}

Jokainen sisätuloavaruus on normiavaruus.

\newtheorem*{schwarz}{Schwarzin ey}
\begin{schwarz}
  $| x \cdot y | \leq \|x\|\|y\|$, missä $x,y \in E$ (sis.tul.ava)
\end{schwarz}

% --------- Metriikka ---------%
\section{Metriikka}

% --------- Avoimet joukot ---------%
\section{Avoimet joukot}
$$U \opn X \iff \forall x \in U \; \exists B(x,r) \subset U \quad (r>0)$$

% --------- Jatkuvuus ---------%
\section{Jatkuvuus}

\begin{definition}
  $f: X \rightarrow Y$ on jva pisteessä $a$, jos $\forall \varepsilon > 0 \;
  \exists \delta > 0$ s.e. $d'(f(x),f(a)) < \varepsilon$ aina kun $x \in X \:
  \land \: d(x,a) < \delta$. Ehto voidaan kirjoittaa muodossa
  $f B(a,\delta) \subset B(f(a), \varepsilon)$ tai $B(a,\delta) \subset
  f^{-1} B(f(a),\epsilon)$.
\end{definition}

\newtheorem*{lipschitz}{Lipschitz-kuvaus}
\begin{lipschitz}
  $f: X \rightarrow Y$ on $M$-Lipschitz (eli myös jva) jos $\exists M \geq 0$ s.e.
  $d'(f(x),f(y)) \leq M d(x,y), \; \forall x,y \in X$.
\end{lipschitz}
Esim. $f: X \rightarrow \mathbb{R}, \: f(x) = d(x,A)$ on $1$-Lipschitz.

\begin{theorem}
  S.E.Y.
  \begin{enumerate}
    \item[(1)]{$f$ jva pisteessä $a$.}
    \item[(2)]{$\forall$ $f(a)$ ystöä $V$ kohti $\exists$ $a$:n ystö U s.e. $f
      U \subset V$.}
    \item[(3)]{$\forall$ $f(a)$ ystöä $V$ kohti $\exists$ $a$:n ystö U s.e. $U
      \subset f^{-1} V$.}
  \end{enumerate}
\end{theorem}

\newtheorem*{alkukuvaehto}{Alkukuvaehto}
\begin{alkukuvaehto}
  $f: X \rightarrow Y$ jva $\iff$ $\forall V \opn Y$: $f^{-1} V \opn X$
\end{alkukuvaehto}

\newtheorem*{rystö}{r-ystö}
\begin{rystö}
  Joukon $\emptyset \neq A \subset X$ $r$-ystö (avoin joukko) on
  $B(A,r) = \{x \in X \; | \; d(x,A) < r\} \quad (r>0)$.
\end{rystö}

\begin{theorem}
  \begin{enumerate}
    \item[(1)]{$X \stackrel{f}{\rightarrow} Y \stackrel{g}{\rightarrow} Z$ jvia
      $\implies$ $g \circ f$ jva}
    \item[(2)]{$f : X \rightarrow \mathbb{R}$ jva ja $f(x) \neq 0$ $\implies$
      $1/f : X \rightarrow \mathbb{R}$ jva}
  \end{enumerate}
\end{theorem}

% --------- Suljetut joukot ---------%
\section{Suljetut joukot}
\newtheorem*{defn:closedSet}{Suljettu joukko}
\begin{defn:closedSet}
  $F \cls X \iff F^C \opn X$.
\end{defn:closedSet}

\newtheorem*{defn:closure}{Sulkeuma}
\begin{defn:closure}
  \begin{align*}
    \bar{A} &=
    \{ x \in X \; | \; \forall \, U(x) \opn X \; \text{pätee} \; U(x) \cap A \neq \emptyset \}
    \\
    &= \{ x \in X \; | \; \forall r > 0 \; \text{pätee} \; B(x,r) \cap A \neq
    \emptyset \}
  \end{align*}
\end{defn:closure}

\begin{theorem}
  Kun kaikki $F_j \cls X$ niin
  \begin{enumerate}
    \item{$\cap_{j \in J} F_j \cls X$.}
    \item{$\cap_{i=1}^{n} F_j \cls X$.}
    \item{$X \cls X$, $\emptyset \cls X$.}
  \end{enumerate}
\end{theorem}

\begin{theorem}
  Jos $U \opn X$ ja $F \cls X$ niin $U \setminus F \opn X$ ja $F \setminus U
  \cls X$.
\end{theorem}

\begin{theorem}
  Olkoot $A,B \subset X$.
  \begin{enumerate}
    \item[(1)]{$A \subset \bar{A}$}
    \item[(2)]{$\bar{A}$ suljettu}
    \item[(3)]{$A \subset B \cls X \implies \bar{A} \subset B$}
    \item[(4)]{$\bar{A}$ on suppein sulj. osajoukko, joka sisältää $A$:n}
    \item[(5)]{$A \subset B \implies \bar{A} \subset \bar{B}$}
    \item[(6)]{$A \cls X \iff A = \bar{A}$}
    \item[(7)]{$\bar{\bar{A}} = \bar{A}$}
    \item[(8)]{$\overline{A \cup B} = \bar{A} \cup \bar{B}$}
    \item[(9)]{$\overline{A \cap B} \subset \bar{A} \cap \bar{B}$}
  \end{enumerate}
\end{theorem}

\begin{theorem}
  Olkoon $\emptyset \neq A \subset R$ ja $\sup A < \infty$. Tällöin $\sup A \in
  \bar{A}$.
\end{theorem}

\begin{theorem}
  $\bar{A} = \{ x \in X \; | \; d(x,A) = 0 \}$ kun $\emptyset \neq A \subset X$.
\end{theorem}

\begin{theorem}
  $f$ jva pisteessä $a$ $\iff$ Jos $A \subset X$ ja $a \in \bar{A}$ niin
  $f(a) \in \overline{fA}$.
\end{theorem}

\begin{theorem}
  S.E.Y.
  \begin{enumerate}
    \item[(1)]{$f$ on jva}
    \item[(2)]{$\forall \, F \cls Y$: $f^{-1}[F] \cls X$ }
    \item[(3)]{$\forall A \subset X$: $f[\bar{A}] \subset \overline{f[A]}$}
  \end{enumerate}
\end{theorem}
\end{multicols*}
\end{document}

% ---- EXAMPLES BELOW ----

% -- Example of figure for imagename.eps
% \begin{figure}
%     \includegraphics[scale=0.5]{imagename}
%     \caption*{Caption be here.}
% \end{figure}

% -- Example of floating figure for imagename.eps
% \begin{floatingfigure}[r]{0.49\textwidth}
%   \includegraphics[scale=0.31]{imagename}
%   \caption*{Caption be here.}
% \end{floatingfigure}

% -- Example of two figures side-by-side
% \begin{figure}[h]
%   \centering
%   \begin{subfigure}{.5\textwidth}
%     \centering
%     \includegraphics[width=.9\linewidth]{fig1.eps}
%     \caption*{Caption here}
%   \end{subfigure}%
%   \begin{subfigure}{.5\textwidth}
%     \centering
%     \includegraphics[width=.9\linewidth]{fig2.eps}
%     \caption*{Caption be here}
%   \end{subfigure}
%   \caption{Shared caption}
%   \label{fig:fighere}
% \end{figure}

% -- Remove pagination
% \thispagestyle{empty}
% \pagestyle{empty}

% -- Example of code listing
%\lstinputlisting[language=Ruby, caption={file.rb}]{./file.rb}

% -- Example of cases-environment
% $$
% f(x) =
% \begin{cases}
%     x^2 & \text{if } x > 0 \\
%     0 & \text{otherwise } \\
% \end{cases}
% $$

% -- Example of multicolumn row in tabular environment
% \multicolumn{3}{|c|}{Cell spanning three columns}
