\documentclass[landscape,a4paper,10pt]{article}

\usepackage[T1]{fontenc}        % cleaner font
\usepackage[utf8]{inputenc}     % UTF-8 support
\usepackage{amsmath}            % more math symbols
\usepackage{amssymb}            % allow math symbols of form \mathbb{...}
\usepackage{graphicx}           % enhanced graphics support
\usepackage{epstopdf}           % automagically turn eps to pdf, for gnuplot
\usepackage{subcaption}         % subfigure support
\usepackage[dvipsnames]{xcolor} % syntax coloring support
\usepackage{listings}           % programming language support
\usepackage{algorithmic}        % pseudocode support
\usepackage{algorithm}
\usepackage{cancel}             % cancel out terms in division
\usepackage{parskip}            % enable spacing between paragraphs
\usepackage{cases}              % enable math function definitions with cases
\usepackage{titling}            % allow adjustment of document title
\usepackage{fullpage}           % 1 inch margins
\usepackage{bm}                 % bold math fonts for vectors and stuff
\usepackage{nicefrac}           % Nicer looking fractions
\usepackage{floatflt}           % Allows text to float around figures / tables
\usepackage{multicol}
\usepackage[landscape]{geometry}
\usepackage{amsthm}
\usepackage{titlesec}

\usepackage[finnish]{babel}    % use Finnish for hyphenation
%\usepackage{hyperref}          % make clickable urls
%\usepackage{color}             % colored text?

% CHEAT SHEET: minimize margins
\geometry{top=0.5cm,left=0.5cm,right=0.5cm,bottom=0.5cm}

% CHEAT SHEET: turn off header and footer
\pagestyle{empty}

% CHEAT SHEET: Redefine section commands to use less space
\makeatletter
\renewcommand{\section}{\@startsection{section}{1}{0mm}%
                                {-1ex plus -.5ex minus -.2ex}%
                                {0.5ex plus .2ex}%x
                                {\color{blue}\normalfont\large\bfseries}}
\renewcommand{\subsection}{\@startsection{subsection}{2}{0mm}%
                                {-1explus -.5ex minus -.2ex}%
                                {0.5ex plus .2ex}%
                                {\normalfont\normalsize\bfseries}}
\renewcommand{\subsubsection}{\@startsection{subsubsection}{3}{0mm}%
                                {-1ex plus -.5ex minus -.2ex}%
                                {1ex plus .2ex}%
                                {\normalfont\small\bfseries}}
\makeatother

% CHEAT SHEET: Don't print section numbers
\setcounter{secnumdepth}{0}

\setlength{\parindent}{0pt}
\setlength{\parskip}{0pt plus 0.5ex}

% Adjust title vertical position
\setlength{\droptitle}{-2cm}

% Set up syntax highlighting for programming languages
\lstloadlanguages{Ruby}
\lstset{%
basicstyle=\ttfamily\bfseries\footnotesize,
commentstyle = \ttfamily\color{orange},
keywordstyle=\ttfamily\color{blue},
stringstyle=\color{red},
showstringspaces=false,
frame=trbl,
}

% Custom probability macros
\def\ci{\perp\!\!\!\perp}              % Independence symbol
\newcommand{\jpr}[2]{P(#1 \, , \, #2)} % Joint probability
\newcommand{\cpr}[2]{P(#1 \, | \, #2)} % Conditional probability


% Avoin joukko / Suljettu joukko symbolit
\newcommand\opn{\mathrel{\ooalign{$\subset$\cr
  \hidewidth\hbox{$\circ\mkern.5mu$}\cr}}}
\newcommand\cls{\mathrel{\ooalign{$\subset$\cr
\hidewidth\raise.225ex\hbox{$\text{{\scriptsize c}}\mkern2mu$}\cr}}}

% Theorems
\newtheoremstyle{customtheoremstyle}{}{}{}{}{\bfseries}{}{ }{}
\theoremstyle{customtheoremstyle}
\newtheorem*{theorem}{Lause}
\newtheorem*{definition}{Määritelmä}
\newtheorem*{example}{Esim.}

\begin{document}
\begin{multicols*}{3}
% CHEAT SHEET: Multicol params

Olkoon $(X,d)$ ja $(Y,d')$ metrisiä avaruuksia.

% --------- Relatiivitopologia ---------%
\section{Relatiivitopologia}

\begin{definition}
  Olkoon $A \subset X$, jolloin saadaan indusoitu M.A. $(A,d_A)$, jonka
  avointen joukkojen kokoelmaa $\mathcal{T}_{d_A} = \{ U : U \opn A \}$
  kutsutaan $A$:n relatiivitopologiaksi.
\end{definition}

% Tarkempi määrittely/kuvaus siitä, mitä suljettu/avoin merkitsee
% relatiivitopologiassa?

\begin{theorem}
  Olkoon $U \subset A \subset X$ ja $E \subset A \subset X$.
  \begin{enumerate}
    \item{$U \opn A \iff \exists V \opn X \;:\; U = A \cap V$.}
    \item{Jos $U \opn A \opn X$, niin $U \opn X$.}
    \item{$\text{cl}_A E = A \cap \bar{E}$.}
    \item{$E \cls A \iff \exists F \cls X \;:\; E = A \cap F$.}
    \item{Jos $E \cls A \cls X$, niin $E \cls X$.}
  \end{enumerate}
\end{theorem}

\begin{theorem}
  Olkoon $A \subset X$. Jos $f: X \rightarrow Y$ jva, niin myös $f|A$ jva.
\end{theorem}

\begin{theorem}
  Olkoon $f: X \rightarrow Y$, $f X \subset B \subset Y$, ja $f_1 : X
  \rightarrow B$ $f$:n määrittelemä kuvaus. Tällöin $f$ jva $\iff$ $f_1$ jva.
\end{theorem}

\begin{theorem}
  Olkoon $X = A_1 \cup ... \cup A_k$, missä $A_j \cls X \;$ kaikilla $j$.
  Olkoon $f: X \rightarrow Y$. Jos $f|A_j$ jva $\forall j$, niin
  $f$ jva.
\end{theorem}


% --------- Sisä-, ulko- ja reunapiste ---------%
\section{Sisä-, ulko- ja reunapiste}
\begin{definition}
  Olkoon $A \subset X$, ja $x \in X$. Sanomme, että $x$ on $A$:n
  \begin{enumerate}
    \item{sisäpiste, jos $x$:llä on ystö $U \subset A$.}
    \item{ulkopiste, jos $x$:llä on ystö $U \subset A^C$. (ei kohtaa $A$:ta)}
    \item{reunapiste, muutoin. Eli $x$:n jokainen ystö kohtaa sekä $A$:n että
      $A^C$:n.}
  \end{enumerate}
  Vastaavat pistejoukot $X = \text{int}(A) \cup \text{ext}(A) \cup \partial A$
  erilliset.
\end{definition}

% Esimerkkejä?

\begin{theorem}
  \begin{enumerate}
    \item{$\text{int}(A) \subset A$, $\text{ext}(A) \subset A^C$.}
    \item{$A$ avoin $\iff$ $\text{int}(A) = A$.}
    \item{$\text{ext}(A) = \text{int}(A^C)$, $\text{int}(A) = \text{ext}(A^C)$.}
    \item{$\text{ext}(A) = \bar{A}^C$,
      $\text{int}(A) = \overline{A^C}^C$.}
    \item{$\bar{A} = \text{int}(A) \cup \partial A = A \cup \partial A$.}
    \item{$\partial A = \bar{A} \cap \overline{A^C} =
      \bar{A} \setminus \text{int}(A)$.}
    \item{$\partial A$ aina suljettu.}
    \item{$\partial A = \partial A^C$.}
    \item{Jos $A$ avoin, niin $\partial A = \bar{A} \setminus A$.}
    \item{$\text{int}(A)$ on $X$:n laajin osajoukko, joka sisältyy $A$:han.}
  \end{enumerate}
\end{theorem}


% --------- Homeomorfismi ---------%
\section{Homeomorfismi}

% ---------- Metriikkojen ekvivalenssi --------%
\section{Metriikkojen ekvivalenssi}

% --------- Jonot ja raja-arvot ---------%
\section{Jonot ja raja-arvot}


% --------- Täydellisyys. Tasainen jatkuvuus ---------%
\section{Täydellisyys. Tasainen jatkuvuus}

% --------- Muuta ---------%
\section{Kompaktius}

\end{multicols*}
\end{document}

% ---- EXAMPLES BELOW ----

% -- Example of figure for imagename.eps
% \begin{figure}
%     \includegraphics[scale=0.5]{imagename}
%     \caption*{Caption be here.}
% \end{figure}

% -- Example of floating figure for imagename.eps
% \begin{floatingfigure}[r]{0.49\textwidth}
%   \includegraphics[scale=0.31]{imagename}
%   \caption*{Caption be here.}
% \end{floatingfigure}

% -- Example of two figures side-by-side
% \begin{figure}[h]
%   \centering
%   \begin{subfigure}{.5\textwidth}
%     \centering
%     \includegraphics[width=.9\linewidth]{fig1.eps}
%     \caption*{Caption here}
%   \end{subfigure}%
%   \begin{subfigure}{.5\textwidth}
%     \centering
%     \includegraphics[width=.9\linewidth]{fig2.eps}
%     \caption*{Caption be here}
%   \end{subfigure}
%   \caption{Shared caption}
%   \label{fig:fighere}
% \end{figure}

% -- Remove pagination
% \thispagestyle{empty}
% \pagestyle{empty}

% -- Example of code listing
%\lstinputlisting[language=Ruby, caption={file.rb}]{./file.rb}

% -- Example of cases-environment
% $$
% f(x) =
% \begin{cases}
%     x^2 & \text{if } x > 0 \\
%     0 & \text{otherwise } \\
% \end{cases}
% $$

% -- Example of multicolumn row in tabular environment
% \multicolumn{3}{|c|}{Cell spanning three columns}
