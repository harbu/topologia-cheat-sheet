\documentclass[landscape,a4paper,10pt]{article}

\usepackage[T1]{fontenc}        % cleaner font
\usepackage[utf8]{inputenc}     % UTF-8 support
\usepackage{amsmath}            % more math symbols
\usepackage{amssymb}            % allow math symbols of form \mathbb{...}
\usepackage{graphicx}           % enhanced graphics support
\usepackage{epstopdf}           % automagically turn eps to pdf, for gnuplot
\usepackage{subcaption}         % subfigure support
\usepackage[dvipsnames]{xcolor} % syntax coloring support
\usepackage{listings}           % programming language support
\usepackage{algorithmic}        % pseudocode support
\usepackage{algorithm}
\usepackage{cancel}             % cancel out terms in division
\usepackage{parskip}            % enable spacing between paragraphs
\usepackage{cases}              % enable math function definitions with cases
\usepackage{titling}            % allow adjustment of document title
\usepackage{fullpage}           % 1 inch margins
\usepackage{bm}                 % bold math fonts for vectors and stuff
\usepackage{nicefrac}           % Nicer looking fractions
\usepackage{floatflt}           % Allows text to float around figures / tables
\usepackage{multicol}
\usepackage[landscape]{geometry}
\usepackage{amsthm}
\usepackage{titlesec}

\usepackage[finnish]{babel}    % use Finnish for hyphenation
%\usepackage{hyperref}          % make clickable urls
%\usepackage{color}             % colored text?

% CHEAT SHEET: minimize margins
\geometry{top=0.5cm,left=0.5cm,right=0.5cm,bottom=0.5cm}

% CHEAT SHEET: turn off header and footer
\pagestyle{empty}

% CHEAT SHEET: Redefine section commands to use less space
\makeatletter
\renewcommand{\section}{\@startsection{section}{1}{0mm}%
                                {-1ex plus -.5ex minus -.2ex}%
                                {0.5ex plus .2ex}%x
                                {\color{blue}\normalfont\large\bfseries}}
\renewcommand{\subsection}{\@startsection{subsection}{2}{0mm}%
                                {-1explus -.5ex minus -.2ex}%
                                {0.5ex plus .2ex}%
                                {\normalfont\normalsize\bfseries}}
\renewcommand{\subsubsection}{\@startsection{subsubsection}{3}{0mm}%
                                {-1ex plus -.5ex minus -.2ex}%
                                {1ex plus .2ex}%
                                {\normalfont\small\bfseries}}
\makeatother

% CHEAT SHEET: Don't print section numbers
\setcounter{secnumdepth}{0}

\setlength{\parindent}{0pt}
\setlength{\parskip}{0pt plus 0.5ex}

% Adjust title vertical position
\setlength{\droptitle}{-2cm}

% Set up syntax highlighting for programming languages
\lstloadlanguages{Ruby}
\lstset{%
basicstyle=\ttfamily\bfseries\footnotesize,
commentstyle = \ttfamily\color{orange},
keywordstyle=\ttfamily\color{blue},
stringstyle=\color{red},
showstringspaces=false,
frame=trbl,
}

% Custom probability macros
\def\ci{\perp\!\!\!\perp}              % Independence symbol
\newcommand{\jpr}[2]{P(#1 \, , \, #2)} % Joint probability
\newcommand{\cpr}[2]{P(#1 \, | \, #2)} % Conditional probability


% Avoin joukko / Suljettu joukko symbolit
\newcommand\opn{\mathrel{\ooalign{$\subset$\cr
  \hidewidth\hbox{$\circ\mkern.5mu$}\cr}}}
\newcommand\cls{\mathrel{\ooalign{$\subset$\cr
\hidewidth\raise.225ex\hbox{$\text{{\scriptsize c}}\mkern2mu$}\cr}}}

% Theorems
\newtheoremstyle{customtheoremstyle}{}{}{}{}{\bfseries}{}{ }{}
\theoremstyle{customtheoremstyle}
\newtheorem*{theorem}{Lause}
\newtheorem*{definition}{Määritelmä}
\newtheorem*{example}{Esim.}

\begin{document}
\begin{multicols*}{3}
% CHEAT SHEET: Multicol params

Olkoon $(X,d)$ ja $(Y,d')$ metrisiä avaruuksia.

% --------- Relatiivitopologia ---------%
\section{Relatiivitopologia}

\begin{definition}
  Olkoon $A \subset X$, jolloin saadaan indusoitu m.a. $(A,d_A)$, jonka
  avointen joukkojen kokoelmaa $\mathcal{T}_{d_A} = \{ U : U \opn A \}$
  kutsutaan $A$:n relatiivitopologiaksi.
\end{definition}

% Tarkempi määrittely/kuvaus siitä, mitä suljettu/avoin merkitsee
% relatiivitopologiassa?

\begin{theorem}
  Olkoon $U \subset A \subset X$ ja $E \subset A \subset X$.
  \begin{enumerate}
    \item{$U \opn A \iff \exists V \opn X \;:\; U = A \cap V$.}
    \item{Jos $U \opn A \opn X$, niin $U \opn X$.}
    \item{$\text{cl}_A E = A \cap \bar{E}$.}
    \item{$E \cls A \iff \exists F \cls X \;:\; E = A \cap F$.}
    \item{Jos $E \cls A \cls X$, niin $E \cls X$.}
  \end{enumerate}
\end{theorem}

\begin{theorem}
  Olkoon $A \subset X$. Jos $f: X \rightarrow Y$ jva, niin myös $f|A$ jva.
\end{theorem}

\begin{theorem}
  Olkoon $f: X \rightarrow Y$, $f X \subset B \subset Y$, ja $f_1 : X
  \rightarrow B$ $f$:n määrittelemä kuvaus. Tällöin $f$ jva $\iff$ $f_1$ jva.
\end{theorem}

\begin{theorem}
  Olkoon $X = A_1 \cup ... \cup A_k$, missä $A_j \cls X \;$ kaikilla $j$.
  Olkoon $f: X \rightarrow Y$. Jos $f|A_j$ jva $\forall j$, niin
  $f$ jva.
\end{theorem}


% --------- Sisä-, ulko- ja reunapiste ---------%
\section{Sisä-, ulko- ja reunapiste}
\begin{definition}
  Olkoon $A \subset X$, ja $x \in X$. Nyt $x$ on $A$:n
  \begin{enumerate}
    \item{sisäpiste, jos $x$:llä on ystö $U \subset A$.}
    \item{ulkopiste, jos $x$:llä on ystö $U \subset A^C$. (ei kohtaa $A$:ta)}
    \item{reunapiste, muutoin. Eli $x$:n jokainen ystö kohtaa sekä $A$:n että
      $A^C$:n.}
  \end{enumerate}
  Vastaavat pistejoukot $X = \text{int}(A) \cup \text{ext}(A) \cup \partial A$
  erilliset.
\end{definition}

% Esimerkkejä?

\begin{theorem}
  \begin{enumerate}
    \item{$\text{int}(A) \subset A$, $\text{ext}(A) \subset A^C$.}
    \item{$A$ avoin $\iff$ $\text{int}(A) = A$.}
    \item{$\text{ext}(A) = \text{int}(A^C)$, $\text{int}(A) = \text{ext}(A^C)$.}
    \item{$\text{ext}(A) = \bar{A}^C$,
      $\text{int}(A) = \overline{A^C}^C$.}
    \item{$\bar{A} = \text{int}(A) \cup \partial A = A \cup \partial A$.}
    \item{$\partial A = \bar{A} \cap \overline{A^C} =
      \bar{A} \setminus \text{int}(A)$.}
    \item{$\partial A$ aina suljettu.}
    \item{$\partial A = \partial A^C$.}
    \item{Jos $A$ avoin, niin $\partial A = \bar{A} \setminus A$.}
    \item{$\text{int}(A)$ on $X$:n laajin osajoukko, joka sisältyy $A$:han.}
  \end{enumerate}
\end{theorem}


% --------- Homeomorfismi ---------%
\section{Homeomorfismi}
\begin{definition}
  Kuvaus $f: X \rightarrow Y$ on homeomorfismi, jos $f$ on bijektio, $f$ jva
  ja, $f^{-1}: Y \rightarrow X$ jva. Tällöin $f: X \approx Y$.
  Vaihtoehto kolmannelle ehdolle: Jos $U \opn X \implies f U \opn Y$, tai
  suljettu-versio (AKE).
\end{definition}

\newtheorem*{defn:upotus}{Upotus}
\begin{defn:upotus}
  on kuvaus $f: X \rightarrow Y$, jos $f_1 : X \rightarrow f X$ homeo, ts. $f$
  on injektio, $f$ jva, ja ${f_1}^{-1}: f X \rightarrow X$ jva (myös AKE).
\end{defn:upotus}

\begin{theorem}
  Homeomorfismi on ekvivalenssirelaatio.
\end{theorem}
% Tarkempi määrittely tarpeen?
% Homeomorfismiluokat?

\begin{theorem}
  Kuvaus $f: X \rightarrow Y$ homeo $\iff$ $f$ bijektio ja $\mathcal{T}_{d'} =
  f \mathcal{T}_d \,(= \{ f U : U \in \mathcal{T}_d \})$.
\end{theorem}

\begin{theorem}
  Homeo. säilyttää kaikki topologiset ominaisuudet.
\end{theorem}
% Tapausket lueteltu?
% Vs. metriset ominaisuudet?

\newtheorem*{defn:bilip}{Bilip}
\begin{defn:bilip}
  Sanomme, että kuvaus $f: X \rightarrow Y$ on $M$-bilipschitz, jos
  $\exists M \geq 1$ s.e.
  $
  d(x,y) / M \leq d'(f(x), f(y)) \leq M d(x,y) \: \forall x,y \in X
  $
\end{defn:bilip}
% Epäyhtälöiden avaaminen; perustelua?

\begin{theorem}
  Bilipschitz-kuvaus on upotus.
\end{theorem}

\newtheorem*{defn:isometria}{Isometria}
\begin{defn:isometria}
  Kuvaus $f: X \rightarrow Y$ on isometria, jos se säilyttää etäisyydet, ts.
  $f$ on 1-bilip. Esim. inkluusiokuvaus $A \hookrightarrow X$.
\end{defn:isometria}

\begin{theorem}
  Bijektiivinen isometria säilyttää kaikki metriset ominaisuudet.
\end{theorem}

% ---------- Metriikkojen ekvivalenssi --------%
\section{Metriikkojen ekvivalenssi}
Joukon $X$ metriikat $d$ ja $e$ ovat topologisesti ekvivalentit, jos
$\mathcal{T}_d = \mathcal{T}_e$, t.s. joukko $U \subset X$ on $d$-avoin joss.
se on $e$-avoin. Merkitsemme $d \sim e$.

\begin{theorem}
  $d \sim e$ $\iff$ $\text{id}: (X,d) \rightarrow (X,e)$ on homeo.
\end{theorem}

\begin{definition}
  Joukon $X$ metriikat $d$ ja $e$ ovat \emph{bilipschitz-ekvivalentit}, jos
  $\text{id}: (X,d) \rightarrow (X,e)$ on bilip.
\end{definition}
% Näennäisesti heikompi ehto messiin?

\begin{theorem}
  $d$ ja $e$ bilipschitz-ekvivalentit $\implies$ $d \sim e$.
\end{theorem}

\begin{theorem}
  Olkoon $d$ joukon $X$ metriikka. Nyt $e(x,y) = \text{min}\{d(x,y), 1\}$ on
  $X$:n metriikka ja $d \sim e$. Siispä jokainen metriikka on ekvivalentti
  rajoitetun metriikan kanssa.
\end{theorem}

% --------- Jonot ja raja-arvot ---------%
\section{Jonot ja raja-arvot}
Joukon $D$ jono $(x_n) \subset D$ on kuvaus $x: \mathbb{N} \rightarrow D$,
jonka arvoja/pisteitä/jäseniä merkitään $x_n = x(n)$.

\newtheorem*{defn:suppeneminen}{Suppeneminen}
\begin{defn:suppeneminen}
  Jono $(x_n) \subset X$ suppenee kohti pistettä $a \in X$, jos $\forall$ $a$:n
  ystöä $U$ kohti $\exists n_0 \in \mathbb{N}$ s.e. $x_n \in U$ kun $n \geq
  n_0$. Merkitään $x_n \rightarrow a$ (kun $n \rightarrow \infty$). Huom. $a$
  yksikäsit.
\end{defn:suppeneminen}

\begin{theorem}
  Olkoon $(x_n) \subset X$ ja $a \in X$. S.E.Y.
  \begin{enumerate}
    \item{$x_n \rightarrow a$.}
    \item{Jos $a \in U$ ystö, niin $x_n \not\in U$ äärellisen monella $n \in
      \mathbb{N}$.}
    \item{$\forall \epsilon > 0 \: \exists n_0 \in \mathbb{N}$ s.e. $d(x_n, a) <
      \epsilon$ kun $n \geq n_0$.}
    \item{$d(x_n, a) \rightarrow 0$.}
  \end{enumerate}
\end{theorem}

\begin{theorem}
  Olkoon $A \subset X$ ja $a \in X$. Tällöin $a \in \bar{A}$ $\iff$ $\exists \:
  (x_n) \subset A$ jolle $x_n \rightarrow a$.
\end{theorem}

\begin{theorem}
  $A \cls X$ $\iff$ $A$ sisältää kaikkien $X$:ssä suppenevien jonojensa
  raja-arvot.
\end{theorem}

\begin{theorem}
  Olkoon $f: X \rightarrow Y$ ja $a \in X$. S.E.Y
  \begin{enumerate}
    \item{$f$ on jatkuva $a$:ssa.}
    \item{Jos $(x_n) \subset X$ jolla $x_n \rightarrow a$, niin $f(x_n)
      \rightarrow f(a)$.}
  \end{enumerate}
\end{theorem}

\begin{theorem}
  Olkoon $(x_k) \subset \mathbb{R}^n$. Nyt $x_k \rightarrow a \in \mathbb{R}^n$
  $\iff$ $\text{pr}_j(x_k) \rightarrow \text{pr}_j(a)$, kun $k \rightarrow
  \infty$, kaikilla $j = 1,...,n$.
\end{theorem}

\begin{theorem}
  Olkoon $(x_n), (y_n) \subset X$ ja $x_n \rightarrow a$, $y_n \rightarrow b$.
  Tällöin $d(x_n, y_n) \rightarrow d(a,b)$.
\end{theorem}

\begin{theorem}
  Olkoon $E$ normiavar., $(x_n), (y_n) \subset E$, ja $x_n \rightarrow x_0$,
  $y_n \rightarrow y_0$. Tällöin $x_n + y_n \rightarrow x_0 + y_0$. Jos $(a_n)
  \subset \mathbb{R}$ ja $a_n \rightarrow a_0$, niin $a_n x_n \rightarrow a_0
  x_0$.
\end{theorem}

\newtheorem*{defn:kasautumisarvo}{Kasautumisarvo}
\begin{defn:kasautumisarvo}
  Piste $a \in X$ on jonon $(x_n) \subset X$ kasautumisarvo, jos $\forall a \in
  U$ ystö kohti, $x_n \in U$ äärettömän monella $n \in \mathbb{N}$.
\end{defn:kasautumisarvo}

\begin{theorem}
  Jos $x_n \rightarrow a$, niin $a$ jonon $(x_n)$ ainoa kasautumisarvo.
\end{theorem}

\begin{theorem}
  Jos $x_n \in A \subset X$ kaikilla $n \in \mathbb{N}$, niin jonon $(x_n)$
  kasautumisarvot kuuluvat sulkeumaan $\bar{A}$.
\end{theorem}

\begin{theorem}
  Olkoon $(x_n) \subset X$. Joukon $A = \{x_n : n \in \mathbb{N} \}$
  kasautumispisteet ovat jonon $(x_n)$ kasautumisarvoja.
\end{theorem}

\newtheorem*{defn:osajono}{Osajono}
\begin{defn:osajono}
  Jonon $(x_n)$ osajono on jono $(y_k)$, jossa $y_k = x_{n_k}$ ja $n_1 < n_2 <
  ...$ aidosti nouseva jono $\mathbb{N}$:ssä.
\end{defn:osajono}

\begin{theorem}
  Jos $x_n \rightarrow a$, niin osajono $y_k \rightarrow a$.
\end{theorem}

\begin{theorem}
  Piste $a$ on jonon $(x_n)$ kas.arvo $\iff$ $(x_n)$:llä osajono joka suppenee
  kohti $a$:ta.
\end{theorem}
% Seuraus: avaruuden X jonolla (x_n) on kas.arvo <=> ainakin yksi suppeneva
% osajono.

\newtheorem*{defn:}{Osajono}
\begin{defn:osajono}
  Jonon $(x_n)$ osajono on jono $(y_k)$, jossa $y_k = x_{n_k}$ ja $n_1 < n_2 <
  ...$ aidosti nouseva jono $\mathbb{N}$:ssä.
\end{defn:osajono}

\newtheorem*{defn:funktiojono}{Funktiojonon suppeneminen}
\begin{defn:funktiojono}
  Olkoon $f_1, f_2, ...$ jono kuvauksia $f_n : D \rightarrow X$, jossa $(X,d)$
  m.a., ja $f: D \rightarrow X$ jokin kuvaus.

  Jono $(f_n)$ suppenee kohti kuvausta $f$ \emph{pisteittäin}
  $D$:ssä, jos $f_n(x) \rightarrow f(x) \; \forall x \in D$. Eli
  $d(f_n(x), f(x)) \rightarrow 0$.

  Jono $(f_n)$ suppenee kohti kuvausta $f$ \emph{tasaisesti} $D$:ssä, jos
  $u_n$ = $\sup_{x \in D} d(f_n(x), f(x)) \rightarrow 0 $.
\end{defn:funktiojono}
% Tasainen suppeneminen => pisteittäin suppeneminen

\begin{theorem}
  $(f_n)$:n jäsenet jvia ja tas.supp. $f$:ää kohti $\implies$ $f$ jva.
\end{theorem}

\newtheorem*{defn:rajaarvot}{Raja-arvot}
\begin{defn:rajaarvot}
  ...
\end{defn:rajaarvot}

% --------- Täydellisyys. Tasainen jatkuvuus ---------%
\section{Täydellisyys. Tasainen jatkuvuus}

% --------- Muuta ---------%
\section{Kompaktius}

\end{multicols*}
\end{document}

% ---- EXAMPLES BELOW ----

% -- Example of figure for imagename.eps
% \begin{figure}
%     \includegraphics[scale=0.5]{imagename}
%     \caption*{Caption be here.}
% \end{figure}

% -- Example of floating figure for imagename.eps
% \begin{floatingfigure}[r]{0.49\textwidth}
%   \includegraphics[scale=0.31]{imagename}
%   \caption*{Caption be here.}
% \end{floatingfigure}

% -- Example of two figures side-by-side
% \begin{figure}[h]
%   \centering
%   \begin{subfigure}{.5\textwidth}
%     \centering
%     \includegraphics[width=.9\linewidth]{fig1.eps}
%     \caption*{Caption here}
%   \end{subfigure}%
%   \begin{subfigure}{.5\textwidth}
%     \centering
%     \includegraphics[width=.9\linewidth]{fig2.eps}
%     \caption*{Caption be here}
%   \end{subfigure}
%   \caption{Shared caption}
%   \label{fig:fighere}
% \end{figure}

% -- Remove pagination
% \thispagestyle{empty}
% \pagestyle{empty}

% -- Example of code listing
%\lstinputlisting[language=Ruby, caption={file.rb}]{./file.rb}

% -- Example of cases-environment
% $$
% f(x) =
% \begin{cases}
%     x^2 & \text{if } x > 0 \\
%     0 & \text{otherwise } \\
% \end{cases}
% $$

% -- Example of multicolumn row in tabular environment
% \multicolumn{3}{|c|}{Cell spanning three columns}
